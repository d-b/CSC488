% ----------------------------------------------------------------
% AMS-LaTeX Paper ************************************************
% **** -----------------------------------------------------------
\documentclass[oneside]{amsart}
\usepackage{graphicx}
\usepackage{color}
\usepackage[letterpaper]{geometry}
\usepackage[colorlinks=false,
            pdfborder={0 0 0},
            pdftitle={CSC488 A5},
            pdfauthor={Daniel Bloemendal},
            pdfsubject={CSC488},
            pdfstartview=FitH,
            pdfmenubar=false,
            pdfdisplaydoctitle=true,
            bookmarks=false]{hyperref}
\usepackage{subcaption}
\usepackage{mathtools}
\usepackage{listings}
\usepackage[table]{xcolor}
% ----------------------------------------------------------------
\vfuzz2pt % Don't report over-full v-boxes if over-edge is small
\hfuzz2pt % Don't report over-full h-boxes if over-edge is small
% THEOREMS -------------------------------------------------------
\newtheorem{thm}{Theorem}[section]
\newtheorem{cor}[thm]{Corollary}
\newtheorem{lem}[thm]{Lemma}
\newtheorem{prop}[thm]{Proposition}
\theoremstyle{definition}
\newtheorem{defn}[thm]{Definition}
\theoremstyle{remark}
\newtheorem{rem}[thm]{Remark}
\numberwithin{equation}{section}
% MATH -----------------------------------------------------------
\newcommand{\norm}[1]{\left\Vert#1\right\Vert}
\newcommand{\abs}[1]{\left\vert#1\right\vert}
\newcommand{\set}[1]{\left\{#1\right\}}
\newcommand{\Real}{\mathbb R}
\newcommand{\eps}{\varepsilon}
\newcommand{\To}{\longrightarrow}
\newcommand{\BX}{\mathbf{B}(X)}
\newcommand{\A}{\mathcal{A}}
\newcommand{\e}{\mathrm{e}}
\newcommand{\AND}{\wedge}
\newcommand{\OR}{\vee}
\newcommand{\NOT}{\neg}
\newcommand{\IMPLIES}{\to}
\newcommand{\TRUE}{\top}
\newcommand{\FALSE}{\bot}
\newcommand{\EQUALS}{\equiv}
\DeclareMathOperator{\sech}{sech}
\newcolumntype{B}{>{\columncolor{black}\color{white}}c}
% ----------------------------------------------------------------
\lstset {
    basicstyle=\fontsize{8}{11}\selectfont\ttfamily,
    frame=none,
    numbers=none
}
% ----------------------------------------------------------------

\begin{document}

\title[CSC488 A5]{CSC488\\ASSIGNMENT 5\\Code Generator}
\author{Daniel Bloemendal}

% ----------------------------------------------------------------
\begin{titlepage}
\maketitle
\thispagestyle{empty}
\tableofcontents
\end{titlepage}
% ----------------------------------------------------------------

\section{Instructions}
\subsection{Bounds checking}
\subsection{Code dump \& syntax highlighting}

\section{Design}
\subsection{Overview}
The overarching theme in the design of the code generator was to avoid exposing the code generator
class \texttt{CodeGen} to the complexities and finer details of the underlying machine. To that
end, an assembler was developed that hides the complexity of addressing code via a label system and
provides an enhanced instruction set, simplifying the emitted code in \texttt{CodeGen}. It should be
noted that the assembler is also entirely decoupled from the rest of the code generator and stands
on its own. The assembler is covered in more detail in \texttt{doc/ASSEMBLER.pdf}. In addition, the
complexities of managing major scopes, their displays, and ensuring that minor scopes are merged
into their enclosing major scopes, is dealt with by the \texttt{Frame} and \texttt{Table} classes.


% ----------------------------------------------------------------
\end{document}
% ----------------------------------------------------------------
