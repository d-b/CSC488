% ----------------------------------------------------------------
% AMS-LaTeX Paper ************************************************
% **** -----------------------------------------------------------
\documentclass[oneside]{amsart}
\usepackage{graphicx}
\usepackage{color}
\usepackage[letterpaper]{geometry}
\usepackage[colorlinks=false,
            pdfborder={0 0 0},
            pdftitle={CSC488 A3},
            pdfauthor={Daniel Bloemendal},
            pdfsubject={CSC488},
            pdfstartview=FitH,
            pdfmenubar=false,
            pdfdisplaydoctitle=true,
            bookmarks=false]{hyperref}
\usepackage{subcaption}
\usepackage{mathtools}
\usepackage{listings}
\usepackage[table]{xcolor}
% ----------------------------------------------------------------
\vfuzz2pt % Don't report over-full v-boxes if over-edge is small
\hfuzz2pt % Don't report over-full h-boxes if over-edge is small
% THEOREMS -------------------------------------------------------
\newtheorem{thm}{Theorem}[section]
\newtheorem{cor}[thm]{Corollary}
\newtheorem{lem}[thm]{Lemma}
\newtheorem{prop}[thm]{Proposition}
\theoremstyle{definition}
\newtheorem{defn}[thm]{Definition}
\theoremstyle{remark}
\newtheorem{rem}[thm]{Remark}
\numberwithin{equation}{section}
% MATH -----------------------------------------------------------
\newcommand{\norm}[1]{\left\Vert#1\right\Vert}
\newcommand{\abs}[1]{\left\vert#1\right\vert}
\newcommand{\set}[1]{\left\{#1\right\}}
\newcommand{\Real}{\mathbb R}
\newcommand{\eps}{\varepsilon}
\newcommand{\To}{\longrightarrow}
\newcommand{\BX}{\mathbf{B}(X)}
\newcommand{\A}{\mathcal{A}}
\newcommand{\e}{\mathrm{e}}
\newcommand{\AND}{\wedge}
\newcommand{\OR}{\vee}
\newcommand{\NOT}{\neg}
\newcommand{\IMPLIES}{\to}
\newcommand{\TRUE}{\top}
\newcommand{\FALSE}{\bot}
\newcommand{\EQUALS}{\equiv}
\DeclareMathOperator{\sech}{sech}
\newcolumntype{B}{>{\columncolor{black}\color{white}}c}
% ----------------------------------------------------------------
\lstset {
    basicstyle=\fontsize{8}{11}\selectfont\ttfamily,
    frame=none,
    numbers=none
}
% ----------------------------------------------------------------

\begin{document}

\title[CSC488 A3]{CSC488\\ASSIGNMENT 3\\Testing}
\author{Daniel Bloemendal}

% ----------------------------------------------------------------
\begin{titlepage}
\maketitle
\thispagestyle{empty}
\tableofcontents
\end{titlepage}
% ----------------------------------------------------------------

\section{Overview}
Our approach to testing was to stress each semantic action. To clarify this, it
should be noted that some semantic actions perform internal book keeping rather
than a check. So we decided to write one or more tests for every semantic action
that involve a semantic check of some kind. Please refer to the
tests/README.tests for a listing of all tests.

\section{Passing tests}
The passing tests are relatively straight forward. We tried to include a variety 
of relatively complicated code stressing many of the features of the language.
Some individual tests for possible corner cases were included as well.
Those tests are \texttt{pass/S10.488}, \texttt{pass/S50.488},
\texttt{pass/S51.488} and \texttt{pass/S52.488}. Please see the comments within
the tests for more information on each test.

\section{Failing tests}
As mentioned in the overview, our approach for failing tests was to include one
or more case for each semantic action involving checks. Each failing tests
includes a comment describing the point of failure and some meta data
\texttt{@line=<line>} or \texttt{@line=[<line\_1>, <line\_2>, ..., <line\_n>]}.
This is important to note because the test runner \texttt{TESTS.py} that I
developed uses this meta data to verify that the test does indeed fail on the
line(s) specified. The test cases are named according to the semantic actions
they fail on. The test runner also uses the filenames to ensure that the
failures do indeed correspond to the correct semantic action.

\section{Test runner}
The test runner \texttt{TESTS.py} is a Python program which tests each passing
and failing test case, and uses the meta data in the failing test cases to
verify that the failures are the expected failures.

% ----------------------------------------------------------------
\end{document}
% ----------------------------------------------------------------
