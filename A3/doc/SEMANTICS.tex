% ----------------------------------------------------------------
% AMS-LaTeX Paper ************************************************
% **** -----------------------------------------------------------
\documentclass[oneside]{amsart}
\usepackage{graphicx}
\usepackage{color}
\usepackage[letterpaper]{geometry}
\usepackage[colorlinks=false,
            pdfborder={0 0 0},
            pdftitle={CSC488 A3},
            pdfauthor={Daniel Bloemendal},
            pdfsubject={CSC488},
            pdfstartview=FitH,
            pdfmenubar=false,
            pdfdisplaydoctitle=true,
            bookmarks=false]{hyperref}
\usepackage{subcaption}
\usepackage{mathtools}
\usepackage{listings}
\usepackage[table]{xcolor}
% ----------------------------------------------------------------
\vfuzz2pt % Don't report over-full v-boxes if over-edge is small
\hfuzz2pt % Don't report over-full h-boxes if over-edge is small
% THEOREMS -------------------------------------------------------
\newtheorem{thm}{Theorem}[section]
\newtheorem{cor}[thm]{Corollary}
\newtheorem{lem}[thm]{Lemma}
\newtheorem{prop}[thm]{Proposition}
\theoremstyle{definition}
\newtheorem{defn}[thm]{Definition}
\theoremstyle{remark}
\newtheorem{rem}[thm]{Remark}
\numberwithin{equation}{section}
% MATH -----------------------------------------------------------
\newcommand{\norm}[1]{\left\Vert#1\right\Vert}
\newcommand{\abs}[1]{\left\vert#1\right\vert}
\newcommand{\set}[1]{\left\{#1\right\}}
\newcommand{\Real}{\mathbb R}
\newcommand{\eps}{\varepsilon}
\newcommand{\To}{\longrightarrow}
\newcommand{\BX}{\mathbf{B}(X)}
\newcommand{\A}{\mathcal{A}}
\newcommand{\e}{\mathrm{e}}
\newcommand{\AND}{\wedge}
\newcommand{\OR}{\vee}
\newcommand{\NOT}{\neg}
\newcommand{\IMPLIES}{\to}
\newcommand{\TRUE}{\top}
\newcommand{\FALSE}{\bot}
\newcommand{\EQUALS}{\equiv}
\DeclareMathOperator{\sech}{sech}
\newcolumntype{B}{>{\columncolor{black}\color{white}}c}
% ----------------------------------------------------------------
\lstset {
    basicstyle=\fontsize{8}{11}\selectfont\ttfamily,
    frame=none,
    numbers=none
}
% ----------------------------------------------------------------

\begin{document}

\title[CSC488 A3]{CSC488\\ASSIGNMENT 3\\Semantic Analysis \& Symbol Table}
\author{Daniel Bloemendal}

% ----------------------------------------------------------------
\begin{titlepage}
\maketitle
\thispagestyle{empty}
\tableofcontents
\end{titlepage}
% ----------------------------------------------------------------

\section{Overview}
While planning the design for semantic analysis, there were a number of
alternative approaches to consider. One suggested approach was to implement all
semantic analysis logic in the AST nodes themselves. Another approach was to
implement some type of visitor pattern, as suggested in,
\texttt{CSC488H1S\_ast.pdf}. Ultimately, I decided to consolidate all semantic
analysis in the single class \texttt{Semantics} in \texttt{Semantics.java}. The
rationale behind this was to avoid a haphazard implementation plan and opt for a
thorough systematic approach. Each semantic action and language construct was
handled one after another in a single file, with code which is both easy to read
and understand. Another critical advantage of the approach is the ease with
which it was possible to verify that the semantic analyzer covers all of the
language grammar and each semantic action by a brief visual inspection of the
code.

\section{Processors}
At its core, the semantic analyzer is based on a depth-first traversal of the
AST. During traversal, as each node is visited a set of \emph{processors} have
the opportunity to inspect the visited node and call the various semantic
actions. Each processor is bound to some target AST class. As a node is visited,
any processor targeting the class of the visited node is fired. Furthermore,
there are two types of processors, \emph{preprocessors} and
\emph{postprocessors}. Preprocessors are fired when an AST node is seen for the
first time. Postprocessors are fired when an AST node is seen for the second and
last time, after all of the node's children have been processed.

The processors are implemented using Java annotations and reflection. A
processor is defined by  annotating a function with the \texttt{@PreProcessor}
or \texttt{@PostProcessor} annotation with the target AST class as an annotation
argument. Below is an example of a processor.

\begin{lstlisting}
/*
 * expression S31 '+' expression S31 S21
 * expression S31 '-' expression S31 S21
 * expression S31 '*' expression S31 S21
 * expression S31 '/' expression S31 S21
 */
@PostProcessor(target = "ArithExpn")
void postArithExpn(BinaryExpn binaryExpn) {
    setTop(binaryExpn.getLeft());
    semanticAction(31); // S31: Check that type of expression or variable is integer.
    setTop(binaryExpn.getRight());
    semanticAction(31); // S31: Check that type of expression or variable is integer.
    semanticAction(21); // S21: Set result type to integer.
}
\end{lstlisting}

At startup, these processors and then stored in a map from AST class name to
function. This is done via reflection in the \texttt{populateMappings} routine
in \texttt{Semantics}. Using reflection, all the methods in the
\texttt{Semantics} class are enumerated and checked for annotations. If our
annotations are found tied to a method the method is then stored in the
appropriate map. The maps are then used during AST traversal to call appropriate
processors as each node is visited.

\section{Actions}

As each processor is executed during AST traversal, those processors in turn
call on semantic actions to perform the necessary semantic checks outlined in
the given semantics documentation, \texttt{CSC488H1S\_semantics.pdf}. Actions
are also implemented using annotations and reflection in a manner almost
identical to processors. This time, an action is defined by declaring an action
function annotated with \texttt{@Action} with the semantic action number as the
argument passed to the annotation. An example of an action is provided below.

\begin{lstlisting}
@Action(number = 46) // Check that lower bound is <= upper bound.
Boolean actionCheckArrayBounds(ArrayDeclPart arrayDecl) {
    ArrayBound b1 = arrayDecl.getBound1(),
               b2 = arrayDecl.getBound2();
    if (arrayDecl.getDimensions() >= 1 && b1.getLowerboundValue() > b1.getUpperboundValue()) {
        setErrorLocation(b1); return false;
    }
    if (arrayDecl.getDimensions() >= 2 && b2.getLowerboundValue() > b2.getUpperboundValue()) {
        setErrorLocation(b2); return false;
    }
    return true;
}
\end{lstlisting}

As previously mentioned, the actions are called on by processors as they operate
on the various nodes of the AST. They are called through the
\texttt{semanticAction} function in \texttt{Semantics}. The
\texttt{semanticAction} function looks up the function corresponding to the
semantic action number and invokes it, passing to it the AST node currently
being visited by the processor. It should be mentioned that processors have the
opportunity to change the target AST node that is being inspected before every
semanticAction call via the \texttt{setTop} function.

The actions have the opportunity to perform either some internal bookkeeping or
a semantic check. If an action performs a semantic check it returns whether the
check succeeded or failed via the return value of the action function. If an
action returns a failure the \texttt{semanticAction} function will report the
failure. It calls on my team mate Peter's pretty printing facilities to
accurately display the location in source code of the failure, by passing it the
AST node where the failure occurred.

\section{Symbol Table}
% ...

% ----------------------------------------------------------------
\end{document}
% ----------------------------------------------------------------
